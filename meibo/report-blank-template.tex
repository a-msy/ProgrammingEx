\documentclass[a4j,11pt]{jarticle}
% ファイル先頭から\begin{document}までの内容(プレアンブル)については,
% 教員からの指示がない限り, { } の中を書き換えるだけでよい.

% ToDo: 提出要領に従って,適切な余白を設定する
\usepackage[top=25truemm,  bottom=30truemm,
            left=25truemm, right=25truemm]{geometry}

% ToDo: 提出要領に従って,適切なタイトル・サブタイトルを設定する
\title{プログラミング演習1レポート \\
       演習課題: 名簿管理プログラムの作成}

% ToDo: 自分自身の氏名と学生番号に書き換える
\author{氏名: 原 直 (HARA, Sunao) \\
        学生番号: 0941xxxx}

% ToDo: 教員の指示に従って適切に書き換える
\date{出題日: 20xx年xx月xx日 \\
      提出日: 20xx年xx月xx日 \\
      締切日: 20xx年xx月xx日 \\}  % 注:最後の\\は不要に見えるが必要.

% ToDo: 図を入れる場合,以下の1行を有効にする
%\usepackage{graphicx}

\begin{document}
\maketitle
% 目次つきの表紙ページにする場合はコメントを外す
%{\footnotesize \tableofcontents \newpage}

%% 本文は以下に書く(\begin{document}~\end{document}が本文である)
%% 課題に応じて適切な章立てを構成すること


%% 本文はここより上に書く(\begin{document}~\end{document}が本文である)
\end{document}
