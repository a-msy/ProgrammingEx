\documentclass[a4j,11pt]{jarticle}
% ファイル先頭から\begin{document}までの内容(プレアンブル)については,
% 教員からの指示がない限り, { } の中を書き換えるだけでよい.

% ToDo: 提出要領に従って,適切な余白を設定する
\usepackage[top=25truemm,  bottom=30truemm,
            left=25truemm, right=25truemm]{geometry}

% ToDo: 提出要領に従って,適切なタイトル・サブタイトルを設定する
\title{プログラミング演習1 \\
       中間レポート}

% ToDo: 自分自身の氏名と学生番号に書き換える
\author{氏名: 今田 将也 (Imada, Masaya) \\
        学生番号: 09430509}

% ToDo: 教員の指示に従って適切に書き換える
\date{出題日: 2019年04月10日 \\
      提出日: 2019年04月18日 \\
      締切日: 2019年05月08日 \\}  % 注:最後の\\は不要に見えるが必要.

% ToDo: 図を入れる場合,以下の1行を有効にする
%\usepackage{graphicx}

\begin{document}
\maketitle
% 目次つきの表紙ページにする場合はコメントを外す
%{\footnotesize \tableofcontents \newpage}

%==================================================================%
\section{はじめに}
プログラミング演習1での名簿管理プログラムの作成課題に取り組むにあたり, 必要な関数を作成した. 

%==================================================================%
\section{作成した関数の説明}

%------------------------------------------------------------------%
\subsection{\texttt{subst}関数}
% 参考:sectionの中では verb が使えない.そんな時は?
 \begin{description}
    \item[関数] \verb|int subst(char *str,char c1,char c2)|
    \item[行数]  4章のソースコードの33行目から43行目に記述してある.
    \item[概要] 引数として与えられた文字列\verb|str|の特定の文字を別の文字に置き換え, 置き換えた文字数をカウントする.
    \item[戻り値]整数型で引数\verb|str|の置き換えた文字数を返す.
    \item[引数]  \verb|char *str|は置き換えたい元の文字列を与える. \verb|char c1|は置き換え対象の文字を与える. \verb|char c2|は置き換え後の文字を与える.
    \item[使用例]
      \begin{verbatim}

      int a = 0;
      printf("before:%s\ncount:%d\n",test,a);
      a=subst(''test'','t','f');
      printf("after:%s\ncount:%d\n",test,a);
      \end{verbatim}
	
  \end{description}
関数\verb|subst|の作成にあたっては,扱う引数が文字列と文字とがあることに注意しなければならない.文字列の場合には引数として\verb|str|の最初の文字をポインタとして与えている
.\\4章の26行目からのwhile文の判定式は,\verb|*str|のみでも問題はないが,後に見返す際に少しでもわかりやすくするためにあえて空文字を指定した.\\関数の役割を区別するために\verb|subst|関数には結果を画面に表示する機能はつけていない.

%------------------------------------------------------------------%
\subsection{\texttt{split}関数}
 \begin{description}
    \item[関数] \verb|int split(char *str,char *ret[],char sep, int max)|
    \item[行数]  4章のソースコードの45行目から59行目に記述してある.
    \item[概要] 引数として与えられた文字列\verb|str|を指定された文字\verb|sep|で文字列\verb|str|の終わりまで順に区切っていく.
    \item[戻り値] 整数型で引数\verb|str|を分割した回数を返す.
    \item[引数]  \verb|char *str|は分割したい元の文字列を与える. \verb|char *ret[]|は分割後の文字列を保存する. \verb|char sep|は区切る対象の文字を与える.\verb|int max|は区切る最大の回数を与える. 
    \item[使用例]
      \begin{verbatim}

          printf("test %s\n",str);
          count=split(str,ret,',',max);
          for(int i = 0; i < count; i++){;
              printf("%d:%s\n",i+1, ret[i]);
          }
          printf("count is %d.\n\n",count);
      \end{verbatim}
\end{description}
関数\verb|split|の内容としては,分割したい文字列のポインタアドレスを分割した文字列を保存する\verb|ret|に格納すしている.そうすることで,元の文字列\verb|str|をコピーしなくとも分割後の文字列を保存することが可能になっている.あなお,後述の\verb|error_split|で分割数が\verb|max|より大きくなった際の処理を行わせている.

%------------------------------------------------------------------%
%------------------------------------------------------------------%
\subsection{\texttt{error\_split}関数}
 \begin{description}
    \item[関数] \verb|void error_split(int check)|
    \item[行数]  4章のソースコードの73行目から87行目に記述してある.
    \item[概要] 引数として与えられた整数値\verb|int check|の値に応じて処理を行う.
    \item[戻り値] なし.
    \item[引数]  \verb|int check|はエラー処理を行わせたい値を与える.
    \item[使用例]
      \begin{verbatim}

          if(count > max)count = -2;
          error_split(count);
      \end{verbatim}
\end{description}
関数\verb|error_split|は関数\verb|split|のエラー処理を,\verb|split|内で行わせるとことで関数の役割が曖昧になることを回避するために作成した.
%------------------------------------------------------------------%
%------------------------------------------------------------------%
\subsection{\texttt{testprint\_split}関数}
 \begin{description}
    \item[関数] \verb|void testprint_split(char *str,char *ret[],char sep,int max)|
    \item[行数]  4章のソースコードの89行目から102行目に記述してある. 
    \item[概要] 関数\verb|split|の内容をテストするための関数.
    \item[戻り値] なし.
    \item[引数]  これらの引数は前述の関数\verb|split|の引数を引き継ぐためのものである.
    \item[使用例]
      \begin{verbatim}

          int max = 5;
          char test1[] = "";
          char test2[] = ",,,";
          char test3[] = ",oka,yama";
          char test4[] = "o,ka,ya,,ma,a";
          char *ret[max];
    
          testprint_split(test1,ret,',',max);
          testprint_split(test2,ret,',',max);
          testprint_split(test3,ret,',',max);
          testprint_split(test4,ret,',',max);
      \end{verbatim}
\end{description}
関数\verb|testprint_split|は関数\verb|split|が正しく目的通りに動作しているかのテストを目的として作成した.\verb|split|内では関数\verb|error_split|同様に関数の役割を区別するために,画面表示機能を持たせてた本関数に分離した.引数を変えるだけでよいため,テストしたい文字列ごとに関数\verb|split|を適用させ,分割後の文字列を表示することを繰り返し記述する必要がなくなる.
%------------------------------------------------------------------%
%\subsection{\texttt{get\_line}関数}%
% 参考:verb以外の場所でアンダーバーを使うには?

%------------------------------------------------------------------%

%==================================================================%
\section{感想}
今回作成した関数\verb|split|は,ポインタにポインタのアドレスを紐づけて文字列を丸々メモリにコピーするような手間が省けるようにしたり,本来必要ではないが,後に変更しやすいよう関数に役割を持たせるために関数を分割するということを意識して取り組めたように思う.前回同様未だにポインタは理解を深められていないところが多いため改めて学習に励みたい.
%==================================================================%
\section{作成したプログラム} \label{sec:sourcecode}
%% プログラムのソースコードは verbatim 環境の中に書く.
%% ここでは,追加で波括弧の内部 {...} だけ文字サイズ等を設定している.
%% 以下の例はフォントサイズ:10pt 行送り:11pt
{\fontsize{10pt}{11pt} \selectfont
\begin{verbatim}
     1  /*
     2   * File:   meibo.c
     3   * Author: 09430509
     4   *
     5   * Created on 2019/04/10
     6   * update on 2019/04/18
     7   */
     8
     9  #include <stdio.h>
    10
    11  int subst(char *str,char c1,char c2);
    12  int split(char *str,char *ret[],char sep,int max);
    13  void testprint_split(char *str,char *ret[],char sep,int max);
    14  void error_split(int check);
    15
    16  int main(void){
    17      int max = 5;
    18      char test1[] = "";//分割したい文字列
    19      char test2[] = ",,,";
    20      char test3[] = ",oka,yama";
    21      char test4[] = "o,ka,ya,,ma,a";
    22      char *ret[max];//分割後に入れる文字配列個数
    23
    24      testprint_split(test1,ret,',',max);
    25      testprint_split(test2,ret,',',max);
    26      testprint_split(test3,ret,',',max);
    27      testprint_split(test4,ret,',',max);
    28
    29      return 0;
    30
    31  }
    32
    33  int subst(char *str,char c1,char c2){
    34      int count = 0;
    35      while(*str != '\0'){
    36          if(*str == c1){
    37              *str = c2;
    38              count++;
    39          }
    40          str++;
    41      }
    42      return count;
    43  }
    44
    45  int split (char *str,char *ret[],char sep,int max){
    46      int count = 0;//分割数
    47
    48      while (1) {
    49          if(*str == '\0') {
    50              break;//からもじなら抜ける
    51          }
    52
    53          ret[count++] = str;//strをいじればretも変わるように分割後の文字列にはポインタを入れる
    54
    55          while( (*str != '\0') && (*str != sep) ){//区切り文字が見つかるまでポインタすすめる
    56              str++;
    57          }
    58
    59          if(*str == '\0') {
    60              break;//区切り文字がなかったら抜ける=文字列はそのまま
    61          }                                                               
    62
    63          *str = '\0';//必ず区切り文字のはずだからくぎる
    64              str++;//インクリメントさせる
    65      }
    66      //if(count<max) count = -1;
    67      if(count>max)count = -2;
    68      error_split(count);
    69
    70      return count;
    71  }
    72
    73  void error_split(int check){
    74      switch(check){
    75          case -1:
    76              printf("luck.\n");
    77              break;
    78
    79          case -2:
    80              printf("over.\n");
    81              break;
    82
    83          default:
    84              break;
    85      }
    86      return;
    87  }
    88
    89  void testprint_split(char *str,char *ret[],char sep,int max){
    90      int count;
    91
    92      printf("test %s\n",str);
    93
    94      count=split(str,ret,',',max);
    95
    96      for(int i = 0; i < count; i++){
    97          printf("%d:%s\n",i+1, ret[i]);
    98      }
    99
   100      printf("count is %d.\n\n",count);
   101      return;
   102  }

\end{verbatim}
}  % <-- !!IMPORTANT!! This line is end of {\fontsize...}
%% 注:行送りの変更は"指定箇所を含む段落”に効果があらわれる.
%%     fontsizeコマンドを用いて,行送りを変える場合は,
%%     その {...} の前後に空白行を入れ,段落を変えるようにすること.
%%     なお,行先頭がコメントから始まる行は空白行とは扱われない.

\end{document}