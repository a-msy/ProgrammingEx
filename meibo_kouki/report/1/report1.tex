\documentclass[a4j,11pt]{jarticle}
% ファイル先頭から\begin{document}までの内容(プレアンブル)については,
% 教員からの指示がない限り, { } の中を書き換えるだけでよい.

% ToDo: 提出要領に従って,適切な余白を設定する
\usepackage[top=25truemm,  bottom=30truemm,
            left=25truemm, right=25truemm]{geometry}

% ToDo: 提出要領に従って,適切なタイトル・サブタイトルを設定する
\title{プログラミング演習1レポート \\
       演習課題: subst関数の作成}

% ToDo: 自分自身の氏名と学生番号に書き換える
\author{氏名: 今田 将也 (Imada, Masaya) \\
        学生番号: 09430509}

% ToDo: 教員の指示に従って適切に書き換える
\date{出題日: 2019年04月10日 \\
      提出日: 2019年04月日 \\
      締切日: 2019年04月17日 \\}  % 注:最後の\\は不要に見えるが必要.

% ToDo: 図を入れる場合,以下の1行を有効にする
%\usepackage{graphicx}

\begin{document}
\maketitle
% 目次つきの表紙ページにする場合はコメントを外す
%{\footnotesize \tableofcontents \newpage}

%% 本文は以下に書く(\begin{document}~\end{document}が本文である)
%% 課題に応じて適切な章立てを構成すること
\section{はじめに}
プログラミング演習1での名簿管理プログラムの作成課題に取り組むにあたり, 必要な\verb|subst|関数を作成した. 
\section{作成した関数の説明}
 \subsection{subst関数}
  \begin{description}
    \item[関数] \verb|int subst(char *str,char c1,char c2)|
    \item[行数]  4章のソースコードの24行目から34行目に記述してある.
    \item[概要] 引数として与えられた文字列\verb|str|の特定の文字を別の文字に置き換え, 置き換えた文字数をカウントする.
    \item[戻り値]整数型で引数\verb|str|の置き換えた文字数を返す.
    \item[引数]  \verb|char *str|は置き換えたい元の文字列を与える. \verb|char c1|は置き換え対象の文字を与える. \verb|char c2|は置き換え後の文字を与える.
    \item[使用例]
      \begin{verbatim}
      int a = 0;
      printf("before:%s\ncount:%d\n",test,a);
      a=subst(''test'','t','f');
      printf("after:%s\ncount:%d\n",test,a);
      \end{verbatim}
	
  \end{description}
関数\verb|subst|の作成にあたっては,扱う引数が文字列と文字とがあることに注意しなければならない.文字列の場合には引数として\verb|str|の最初の文字をポインタとして与えている
.\\4章の26行目からのwhile文の判定式は,\verb|*str|のみでも問題はないが,後に見返す際に少しでもわかりやすくするためにあえて空文字を指定した.\\関数の役割を区別するために\verb|subst|関数には画面に表示する機能はつけていない.
\section{感想}
文字と文字列の扱い方の違いや文字列操作について改めて理解をすることができた.ポインタの仕組みについてはあまり理解できていないところがあるため今一度教科書を読み返して今後苦労することが少なくなるようにしておきたい.
\section{作成したプログラム}
\begin{verbatim}
     1	/* 
     2	 * File:   main.c
     3	 * Author: 09430509
     4	 *
     5	 * Created on 2019/04/10
     6	 * update on 2019/04/10
     7	 */
     8	
     9	#include <stdio.h>
    10	
    11	int subst(char *str,char c1,char c2);
    12	
    13	int main(void) {
    14	    char test[] = "aabbcc";
    15	    int a=0;
    16	    
    17	    printf("before:%s\ncount:%d\n",test,a);
    18	    a=subst(test,'a','f');
    19	    printf("after:%s\ncount:%d\n",test,a);
    20	    
    21	    return 0;
    22	}
    23	
    24	int subst(char *str,char c1,char c2){
    25	    int count = 0;//文字数のカウント
    26	    while(*str != '\0'){
    27	        if(*str == c1){//変換対象があれば
    28	            *str = c2;//変換
    29	            count++;//文字数カウント
    30	        }
    31	        str++;//ポインタインクリメント
    32	    }
    33	    return count;
    34	}
    35	

\end{verbatim}
%% 本文はここより上に書く(\begin{document}~\end{document}が本文である)
\end{document}

